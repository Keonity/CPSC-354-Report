\documentclass{article}

\usepackage{amsthm}
\usepackage{amsfonts}
\usepackage{amsmath}
\usepackage{amssymb}
\usepackage{fullpage}
\usepackage[usenames]{color}
\usepackage{hyperref}
  \hypersetup{
    colorlinks = true,
    urlcolor = blue,       % color of external links using \href
    linkcolor= blue,       % color of internal links 
    citecolor= blue,       % color of links to bibliography
    filecolor= blue,        % color of file links
    }
    
\usepackage{listings}

\definecolor{dkgreen}{rgb}{0,0.6,0}
\definecolor{gray}{rgb}{0.5,0.5,0.5}
\definecolor{mauve}{rgb}{0.58,0,0.82}

\lstset{frame=tb,
  language=haskell,
  aboveskip=3mm,
  belowskip=3mm,
  showstringspaces=false,
  columns=flexible,
  basicstyle={\small\ttfamily},
  numbers=none,
  numberstyle=\tiny\color{gray},
  keywordstyle=\color{blue},
  commentstyle=\color{dkgreen},
  stringstyle=\color{mauve},
  breaklines=true,
  breakatwhitespace=true,
  tabsize=3
}

\theoremstyle{theorem} 
   \newtheorem{theorem}{Theorem}[section]
   \newtheorem{corollary}[theorem]{Corollary}
   \newtheorem{lemma}[theorem]{Lemma}
   \newtheorem{proposition}[theorem]{Proposition}
\theoremstyle{definition}
   \newtheorem{definition}[theorem]{Definition}
   \newtheorem{example}[theorem]{Example}
\theoremstyle{remark}    
  \newtheorem{remark}[theorem]{Remark}


\title{CPSC-354 Report}
\author{Keoni Lanoza  \\ Chapman University}

\date{\today}

\begin{document}

\maketitle

\begin{abstract}
Short  summary of purpose and content.  
\end{abstract}

\tableofcontents

\section{Introduction}\label{intro}

Replace this entire Section~\ref{intro} with your own short introduction. 

\subsection{General Remarks}

First you need to \href{https://www.latex-project.org/get/}{download and install} LaTeX.\footnote{Links are typeset in blue, but you can change the layout and color of the links if you locate the  \texttt{hypersetup} command.}
%
For quick experimentation, you can use an online editor such as \href{https://www.overleaf.com/learn}{Overleaf}. But to grade the report I will used the time-stamped pdf-files in your git repository.  

 
\medskip\noindent
LaTeX is a markup language (as is, for example, HTML). The source code is in a \verb+.tex+ file and needs to be compiled for viewing, usually to \verb+.pdf+.


\medskip\noindent
If you want to change the default layout, you need to type commands. For example, \verb+\medskip+ inserts a medium vertical space and \verb+\noindent+ starts a paragraph without indentation.
 
\medskip\noindent
Mathematics is typeset between double dollars, for example $$x+y=y+x.$$


\subsection{LaTeX Resources}

I start a new subsection, so that you can see how it appears in the table of contents.

\subsubsection{Subsubsections}

Sometimes it is good to have subsubsections.

\subsubsection{Itemize and enumerate}

\begin{itemize}
\item This is how you itemize in LaTeX.
\item I think a good way to learn LaTeX is by starting from this template file and build it up step by step. Often stackoverflow will answer your questions. But here are a few resources:
  \begin{enumerate}
  \item \href{https://www.overleaf.com/learn/latex/Learn_LaTeX_in_30_minutes}{Learn LaTeX in 30 minutes}
  \item \href{https://www.latex-project.org/}{LaTeX – A document preparation system}\end{enumerate}
\end{itemize}

\subsubsection{Typesetting Code}

A typical project will involve code. For the example below I took the LaTeX code from \href{https://stackoverflow.com/a/3175141/4600290}{stackoverflow} and the Haskell code from \href{https://hackmd.io/@alexhkurz/HylLKujCP}{my tutorial}.

\begin{lstlisting}
-- run the transition function on a word and a state
run :: (State -> Char -> State) -> State -> [Char] -> State
run delta q [] = q
run delta q (c:cs) = run delta (delta q c) cs 
\end{lstlisting}
%
Short snippets such as \texttt{run :: (State -> Char -> State) -> State -> [Char] -> State} can also be directly fitted into text. There are several ways of doing this, for example, \verb@run :: (State -> Char -> State) -> State -> [Char] -> State@ is slightly different in terms of spaces and linebreaking (and can lead to layout that is better avoided), as is
\begin{verbatim}run :: (State -> Char -> State) -> State -> [Char] -> State\end{verbatim}

\noindent
For more on the topic see \href{https://www.overleaf.com/latex/examples/code-presentations-example-different-ways-shown-in-beamer-metropolis/tsxpnyjbhbds}{Code-Presentations Example}.

\medskip\noindent
Generally speaking,  the methods for displaying code discussed above work well only for short listings of code. For entire programs, it is better to have external links to, for example, Github or \href{https://replit.com/@alexhkurz/automata01#main.hs}{Replit} (click on the "Run" button and/or the ``Code" tab).

\subsubsection{More Mathematics}

We have already seen $x+y=y+x$ as an example of inline maths. We can also typeset mathematics in display mode, for example
$$\frac x y =\frac{xy}{y^2},$$

\noindent
Here is an example of equational reasoning that spans several lines:
\begin{align*}
{\rm fib}(3)
& = {\rm fib}(1) +{\rm fib}(2)  & {\rm fib}(n+2) = {\rm fib}(n)  + {\rm fib}(n+1) \\
& = {\rm fib}(1) +{\rm fib}(0)  + {\rm fib}(1) & {\rm fib}(n+2) = {\rm fib}(n)  + {\rm fib}(n+1) \\
& = 1 + 0  +1 & {\rm fib}(0) = 0,   {\rm fib}(1) = 1\\
& = 2 & {\rm arithmetic}
\end{align*}

\subsubsection{Definitons, Examples, Theorems, Etc}

\begin{definition} 
This is a definition.
\end{definition}

\begin{example}
This is an example.
\end{example}

\begin{proposition}
This is a proposition.
\end{proposition}

\begin{theorem}
This is a theorem.
\end{theorem}

\noindent You can also create your own environment, eg if you want to have Question, Notation, Conjecture, etc.

\subsection{Plagiarism}

To avoid plagiarism, make sure that in addition to \cite{PL} you also cite all the external sources you use. Make sure you cite all your references in your text, not only at the end.

\section{Homework}\label{homework}

This section will contain your solutions to homework.
For every week, you will have a subsection that contains your answers.

\subsection{Week 1}

\begin{verbatim}
print("Please enter integer A")
aString = input("a: ")
while True:
    try: 
        a = int(aString)
        break
    except ValueError:
        print("Not an integer. Please try again.")
        print("Please enter INTEGER A")
        aString = input("a: ")
    
print("Please enter integer B")
bString = input("b: ")
while True:
    try: 
        b = int(bString)
        break
    except ValueError:
        print("Not an integer. Please try again.")
        print("Please enter INTEGER B")
        bString = input("b: ")

a = int(aString)
b = int(bString)

while (a != b):
    if (a > b):
        a = a - b
    elif (b > a):
        b = b - a
        
print ("The greatest common divisor is: " + str(a))
\end{verbatim}

\indent
This program works by first asking the user to input an integer named "A". 
The program stores this input in a variable called "aString" and then utilizes error checking to make sure the user's input can be converted into an integer.
If the input cannot be converted into an integer, the program loops until the user enters valid input.
If the user's input passes error checking, the program then prompts the user for an integer named "B" and follows the same error-checking process.
Once the user passes error checking for both variables, the program enters a loop.
In the loop, if integer A is greater than integer B, then integer A is replaced with the value of integer A - integer B.
If integer B is greater than integer A, then integer B is replaced with the value of integer B - integer A.
This process repeats until integer A is equal to integer B. When this is reached, the greatest common divisor of the original integer A and original integer B is printed out to the user.

For example, if we took A to be an integer representing the value of 9, and B to be an integer representing the value of 33, the program would follow this process:
Because B, which equals 33 is greater than A, which equals 9. B's value would be replaced with B - A, which is 24. B is still greater than A, so B's value would be replaced by B - A again which now equals 15.
15 is still greater than 9, so B would become 6. Now, A with a value of 9 is greater than B which has a value of 6. A would be replaced with A - B which equals 3. This makes B greater than A again.
B is now replaced with B - A, or 6 - 3, which equals 3. Now that A and B are equal, the greatest common divisor, which is 3 because both A and B equal 3, is output to the user.

\subsection{Week 2}

\ldots

\section{Project}

Introductory remarks ...

The following structure should be suitable for most practical projects. 

\subsection{Specification}
\subsection{Prototype}
\subsection{Documentation}
\subsection{Critical Appraisal}

\ldots

\section{Conclusions}\label{conclusions}

(approx 400 words)

In the conclusion, I want a critical reflection on the content of the course. Step back from the technical details. How does the course fit into the wider world of programming languages and software engineering?

\begin{thebibliography}{99}
\bibitem[PL]{PL} \href{https://github.com/alexhkurz/programming-languages-2022/blob/main/README.md}{Programming Languages 2022}, Chapman University, 2022.
\end{thebibliography}

\end{document}
